\section{Arctangent}

\subsection{$\myarctan{x} + \myarctan{\nicefrac{1}{x}}$}
\subsubsection{over Triangle}
Note: Not my observation, nor my explanation, but I do love this\\ \\
Consider the following right angled triangle $\triangle ABC$\\
\begin{center}
\begin{tikzpicture}[thick]
    \coordinate (A) at (0,0) node[anchor=north east] at (A) {$A$};
    \coordinate (B) at (3,0) node[anchor=north west] at (B) {$B$};
    \coordinate (C) at (3,3) node[anchor=south west] at (C) {$C$};
    
    \draw (A) -- (B) node [midway, below] {$1$}; %from A to B, with 1 under the line
    \draw (B) -- (C) node [midway, right] {$x$}; %from B to C, with x to the right

    \draw (A) -- (C)
    pic[angle radius=4mm, draw=black, "."] {angle = C--B--A}
    pic[angle radius=9mm, draw=black, "$\alpha$"] {angle = B--A--C}
    pic[angle radius=9mm, draw=black, "$\beta$"] {angle = A--C--B};
\end{tikzpicture}
\end{center}

All angles in a triangle sum up to $\pi \implies \alpha + \beta = \frac{\pi}{2}$
\begin{gather}
\begin{aligned}
    \tan(\alpha) = x &\Leftrightarrow \alpha = \arctan(x)\\
    \tan(\beta) = \frac{1}{x} &\Leftrightarrow \beta = \myarctan{\frac{1}{x}}
\end{aligned}\\
\alpha+\beta = \arctan(x) + \myarctan{\frac{1}{x}} = \frac{\pi}{2}
\end{gather}

From symmetry we know that $\arctan(-x) = - \arctan(x)$
\begin{align}
    \therefore\ \arctan(-x) + \myarctan{-\frac{1}{x}} &= 
                -\bracket{\arctan(x)+\myarctan{\frac{1}{x}}} \\
    &= -\frac{\pi}{2} \\
\end{align}
%
\begin{align}
\result[123]{ 
    \arctan(x)+\myarctan{\frac{1}{x}} = 
    {\begin{cases}
            \hfil \frac{\pi}{2} &\mbox{if } x>0 \\
            -\frac{\pi}{2} &\mbox{if } x<0
    \end{cases}}} \\
\result[a]{\arctan(x)+\myarctan{\frac{1}{x}} = \sgn(x) \cdot \frac{\pi}{2}}
\end{align}

\subsubsection{over $\sin$ and $\cos$}
\paragraph{option 1:}\ \\
Note: observation by D. J.
\begin{gather}
\boxed{x,\alpha,\beta >0}\\
\alpha+\beta= \frac{\pi}{2} \implies \beta=\frac{\pi}{2}-\alpha \\
\mbox{let } \tan(\alpha)=\frac{\sin(\alpha)}{\cos(\alpha)}=x \\
\tan(\beta)=\mytan{\frac{\pi}{2}-\alpha} = \frac{\mysin{\frac{\pi}{2}-\alpha}}{\mycos{\frac{\pi}{2}-\alpha}}
    = \frac{\cos(\alpha)}{\sin(\alpha)}=\frac{1}{\tan(\alpha)}=\frac{1}{x}\\
\implies \alpha = \arctan(x), \quad \beta = \frac{\pi}{2} - \alpha = \myarctan{\frac{1}{x}}\\
\implies \arctan(x) + \myarctan{\frac{1}{x}} = \alpha + \frac{\pi}{2} -\alpha = \frac{\pi}{2}\\
\tan(-\alpha)=-\tan(\alpha)=-x\\
\tan(-\beta)=-\tan(\beta)=-\frac{1}{x}\\
\implies \arctan(-x) =  -\alpha,\quad  \myarctan{-\frac{1}{x}}=-\beta = -\frac{\pi}{2} + \alpha\\
\implies \arctan(-x) + \myarctan{-\frac{1}{x}} = - \alpha - \frac{\pi}{2} + \alpha = -\frac{\pi}{2} \\
\\
\result{\arctan(x) + \myarctan{\frac{1}{x}} = 
    {\begin{cases}
    \hfil \frac{\pi}{2} &\mbox{für } x>0 \\ %hfil for align middle
   - \frac{\pi}{2} &\mbox{für } x<0
    \end{cases}}
}
\end{gather}

\paragraph{option 2:}\ \\
Note: observation by D. J.
\begin{gather}
\myarctan x + \myarctan{\frac{1}{x}} = \ ? \\
\mytan\alpha = \frac{\mysin\alpha}{\mycos\alpha} = x\\
\mytan{\beta} = \frac{1}{x} = \frac{1}{\mytan\alpha} = \frac{\mycos{\alpha}}{\mysin{\alpha}}
 = \frac{\mysin{\frac{\pi}{2}-\alpha}}{\mycos{\frac{\pi}{2}-\alpha}}
 = \frac{\mysin{\beta}}{\mycos{\beta}} \\
 \implies \beta = \frac{\pi}{2} - \alpha \implies \alpha + \beta = \frac{\pi}{2} \\
 \implies \mytan{\beta} = \mytan{\frac{\pi}{2} - \alpha} = \frac{1}{\mytan{\alpha}} = \frac{1}{x} \\
 \implies  \myarctan{x} + \myarctan{\frac{1}{x}} = \cancel{\alpha} + \frac{\pi}{2} - \cancel{\alpha} = \frac{\pi}{2} \\
\end{gather}
%
From symmetry we know that $\myarctan{-x} = - \myarctan{x}$
\begin{gather}
    \begin{aligned}
    \therefore\ \myarctan{-x} + \myarctan{-\frac{1}{x}} &= 
                -\bracket{\myarctan{x}+\myarctan{\frac{1}{x}}} \\
    &= -\frac{\pi}{2} 
    \end{aligned} \\
\result{ 
    \myarctan{x}+\myarctan{\frac{1}{x}} = 
    {\begin{cases}
            \hfil \frac{\pi}{2} &\mbox{if } x>0 \\
            -\frac{\pi}{2} &\mbox{if } x<0
    \end{cases}}}
\end{gather}


\subsubsection{over derivative and limit}
%
\begin{gather}
f(x) = \myarctan{x} + \myarctan{\frac{1}{x}}\quad D_f=\R\setminus \{0\}\\
f'(x) = \frac{1}{1+x^2} + \frac{1}{1+\bracket{\dfrac{1}{x}}^2} \cdot \bracket{-\frac{1}{x^2}}
= \frac{1}{1+x^2} - \frac{1}{x^2 + 1} = 0
\end{gather}
If $f'(x)=0$ then f(x) has at least one constant value\\
Since 0 is excluded, we should check for values above, and below zero, because, this could be the only
jump in the function, because it's continuous everywhere else (in $\R^+$ and in $\R^-$):

Case 1: $x>0$
\[
    f(1) = \myarctan{1} + \myarctan{\frac{1}{1}} = \frac{\pi}{4} + \frac{\pi}{4} = \frac{\pi}{2}
\]

Case 2: $x<0$
\[
    f(-1) = \myarctan{-1} + \myarctan{\frac{1}{-1}} = -\frac{\pi}{4} - \frac{\pi}{4} = -\frac{\pi}{2}
\]

\[
\result{f(x) = {\begin{cases} \hfil \frac{\pi}{2} &\mbox{für }x>0 \\ -\frac{\pi}{2} &\mbox{für } x<0\end{cases}}}
\]