%fixing spacing after around \left \right
\NewCommandCopy{\leftold}{\left}
\NewCommandCopy{\rightold}{\right}
\renewcommand{\left}{\mathopen{}\mathclose\bgroup\leftold}
\renewcommand{\right}{\aftergroup\egroup\rightold}

%trig functions
\newcommand{\mytan}[1]{\tan\bracket{#1}}
\newcommand{\myarctan}[1]{\arctan\bracket{#1}}
\newcommand{\mysin}[1]{\sin\bracket{#1}}
\newcommand{\myarcsin}[1]{\arcsin\bracket{#1}}
\newcommand{\mycos}[1]{\cos\bracket{#1}}
\newcommand{\myarccos}[1]{\arccos\bracket{#1}}

%math commands
\newcommand{\bracket}[1]{\left(#1\right)}
\newcommand{\abs}[1]{\left|#1\right|}

%number sets
\newcommand{\N}{\mathbb{N}}
\newcommand{\Z}{\mathbb{Z}}
\newcommand{\Q}{\mathbb{Q}}
\newcommand{\R}{\mathbb{R}}
\newcommand{\I}{\mathbb{I}}
\newcommand{\C}{\mathbb{C}}


%for result
%Using Aboxed, because of it's ability to align with the outer align env
\makeatletter
\newcommand{\colorboxed}[1]{\fcolorbox{black}{black!0}{\m@th$\displaystyle#1$}}
\xpatchcmd{\@Aboxed}{\boxed}{\colorboxed}{}{}
\makeatother


\makeatletter
\newcommand{\resultnoimplies}[1]{
    \ifthenelse{\equal{\@currenvir}{align}} %if in align env
    {\Aboxed{#1}} %don't use aligned, to avoid errors
    {\begin{aligned}\Aboxed{#1}\end{aligned}} %otherwise do
}
\makeatother
%for writing an aligned env inside you need to put that in curly braces (definitely works for cases, don't know about the others)
\newcommand{\result}[2][noalign]{ %with optional arg, default being noalign
\ifthenelse{\equal{#1}{noalign}} %without optional arg
    {\implies \resultnoimplies{#2}} % don't align
    {\implies \resultnoimplies{&#2}} % with do
}


%signum function
\DeclareMathOperator{\sgn}{sgn}

%complex numbers
\DeclareMathOperator{\conjugate}{conj}
\newcommand{\conj}[1]{z^{*}} % \bar{z} or z^{*}
\renewcommand{\i}{i}

%changing the look of the Re and Im function
\ifbool{fractureCommands}
{
    \NewCommandCopy{\Reold}{\Re} %for use later
    \NewCommandCopy{\Imold}{\Im}
}
{
    \newcommand{\Reold}{\operatorname{Re}} % if fracture isn't wanted
    \newcommand{\Imold}{\operatorname{Im}} %if fracture isn't wanted
}

%defines the commmands for real and imaginary part of a complex number, it makes symbol, parentesis, and the argument
%for use of function without parentesis, use Reold/Imold
\renewcommand{\Re}[1]{\Reold\left(#1\right)}
\renewcommand{\Im}[1]{\Imold\left(#1\right)}
