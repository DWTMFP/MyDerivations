%Math operators
\DeclareMathOperator{\sgn}{sgn}
\DeclareMathOperator{\conjugate}{conj}


%fixing spacing after around \left \right
\NewCommandCopy{\leftold}{\left}
\NewCommandCopy{\rightold}{\right}
\renewcommand{\left}{\mathopen{}\mathclose\bgroup\leftold}
\renewcommand{\right}{\aftergroup\egroup\rightold}

%trig and other functions
\newcommand{\mytan}[2][]{\tan\bracket[#1]{#2}}
\newcommand{\myarctan}[2][]{\arctan\bracket[#1]{#2}}
\newcommand{\mysin}[2][]{\sin\bracket[#1]{#2}}
\newcommand{\myarcsin}[2][]{\arcsin\bracket[#1]{#2}}
\newcommand{\mycos}[2][]{\cos\bracket[#1]{#2}}
\newcommand{\myarccos}[2][]{\arccos\bracket[#1]{#2}}
\newcommand{\myln}[2][]{\ln\bracket[#1]{#2}}


\newcommand{\abs}[1]{\left|#1\right|}
\newcommand{\floor}[1]{\left\lfloor#1\right\rfloor}
\newcommand{\ceil}[1]{\left\lceil#1\right\rceil}
\newcommand{\sgnfn}[1]{\sgn\bracket{#1}}

%math commands
\newcommand{\bracket}[2][]{
    \ifthenelse{\equal{#1}{s}} %if optional arg = s
    {\left[#2\right]} %use square Brackets
    {\ifthenelse{\equal{#1}{c}} % if optional arg = c
        {\left\{#2\right\}} % use curly braces
        {\ifthenelse{\equal{#1}{a}} %if optional arg = a
        {\abs{#1}}          % use abs
        {\left(#2\right)}}}    %otherwise (no / unvalid optional arg) use normal brackets
}

%number sets
\newcommand{\N}{\mathbb{N}}
\newcommand{\Z}{\mathbb{Z}}
\newcommand{\Q}{\mathbb{Q}}
\newcommand{\R}{\mathbb{R}}
\newcommand{\I}{\mathbb{I}}
\newcommand{\C}{\mathbb{C}}

\newcommand{\e}{\ensuremath{\mathrm{e}}}


%Result Command
% for writing an aligned env inside you need to put that in curly braces 
% (definitely works for cases, don't know about the others)
\makeatletter
%Using Aboxed, because of it's ability to align with the outer align env
\newcommand{\colorboxed}[1]{\fcolorbox{black}{black!0}{\m@th$\displaystyle#1$}}
\xpatchcmd{\@Aboxed}{\boxed}{\colorboxed}{}{}

%Result Command Definition
\newcommand{\@noalign}{noalign} %just there to avoid typos in the result command definitions
\newcommand{\@resultBoxCommand}[1]{\Aboxed{#1}} %for easily changing the command, which creates the box
\newcommand{\@impliesresult}{\implies} %for theoretically switching to something else, or for removing the arrow

%This Command just checks if the result is inside an align env and then puts the arg inside a box
\newcommand{\@resultBaseCommand}[1]{
    \ifthenelse{\equal{\@currenvir}{align}} %if in align env
    {\@resultBoxCommand{#1}} %don't use aligned, to avoid errors
    {\ifthenelse{\equal{\@currenvir}{aligned}} %same for aligned
    {\@resultBoxCommand{#1}} 
    {\begin{aligned}\Aboxed{#1}\end{aligned}}} %otherwise do
}

%The result without the Box, also decides, if it aligns with the outer stuff or not
\newcommand{\resultnoimplies}[2][\@noalign]{
    \ifthenelse{\equal{#1}{\@noalign}} %without optional arg
    {\@resultBaseCommand{#2}} % don't align
    {\@resultBaseCommand{&#2}} % with arg do
}
% The normal result command with implies
\newcommand{\result}[2][\@noalign]{\@impliesresult \resultnoimplies[#1]{#2}} %with optional arg, default being noalign
\makeatother

%complex numbers
\newcommand{\conj}[1]{z^{*}} % \bar{z} or z^{*}
\renewcommand{\i}{i}

%changing the look of the Re and Im function
\ifbool{fractureCommands}
{
    \NewCommandCopy{\Reold}{\Re} %for use later
    \NewCommandCopy{\Imold}{\Im}
}
{
    \newcommand{\Reold}{\operatorname{Re}} % if fracture isn't wanted
    \newcommand{\Imold}{\operatorname{Im}} %if fracture isn't wanted
}

%defines the commmands for real and imaginary part of a complex number, it makes symbol, parentesis, and the argument
%for use of function without parentesis, use Reold/Imold
\renewcommand{\Re}[1]{\Reold\left(#1\right)}
\renewcommand{\Im}[1]{\Imold\left(#1\right)}
